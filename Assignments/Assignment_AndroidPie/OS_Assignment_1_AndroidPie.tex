%%%%%%%%%%%%  Generated using docx2latex.com  %%%%%%%%%%%%%%

%%%%%%%%%%%%  v2.0.0-beta  %%%%%%%%%%%%%%

\documentclass[12pt]{article}
\usepackage{amsmath}
\usepackage{latexsym}
\usepackage{amsfonts}
\usepackage[normalem]{ulem}
\usepackage{array}
\usepackage{amssymb}
\usepackage{graphicx}
\usepackage[backend=biber,
style=numeric,
sorting=none,
isbn=false,
doi=false,
url=false,
]{biblatex}\addbibresource{bibliography.bib}

\usepackage{subfig}
\usepackage{wrapfig}
\usepackage{wasysym}
\usepackage{enumitem}
\usepackage{adjustbox}
\usepackage{ragged2e}
\usepackage[svgnames,table]{xcolor}
\usepackage{tikz}
\usepackage{longtable}
\usepackage{changepage}
\usepackage{setspace}
\usepackage{hhline}
\usepackage{multicol}
\usepackage{tabto}
\usepackage{float}
\usepackage{multirow}
\usepackage{makecell}
\usepackage{fancyhdr}
\usepackage[toc,page]{appendix}
\usepackage[hidelinks]{hyperref}
\usetikzlibrary{shapes.symbols,shapes.geometric,shadows,arrows.meta}
\tikzset{>={Latex[width=1.5mm,length=2mm]}}
\usepackage{flowchart}\usepackage[paperheight=11.0in,paperwidth=8.5in,left=1.0in,right=1.0in,top=1.0in,bottom=1.0in,headheight=1in]{geometry}
\usepackage[utf8]{inputenc}
\usepackage[T1]{fontenc}
\TabPositions{0.5in,1.0in,1.5in,2.0in,2.5in,3.0in,3.5in,4.0in,4.5in,5.0in,5.5in,6.0in,}

\urlstyle{same}


 %%%%%%%%%%%%  Set Depths for Sections  %%%%%%%%%%%%%%

% 1) Section
% 1.1) SubSection
% 1.1.1) SubSubSection
% 1.1.1.1) Paragraph
% 1.1.1.1.1) Subparagraph


\setcounter{tocdepth}{5}
\setcounter{secnumdepth}{5}


 %%%%%%%%%%%%  Set Depths for Nested Lists created by \begin{enumerate}  %%%%%%%%%%%%%%


\setlistdepth{9}
\renewlist{enumerate}{enumerate}{9}
		\setlist[enumerate,1]{label=\arabic*)}
		\setlist[enumerate,2]{label=\alph*)}
		\setlist[enumerate,3]{label=(\roman*)}
		\setlist[enumerate,4]{label=(\arabic*)}
		\setlist[enumerate,5]{label=(\Alph*)}
		\setlist[enumerate,6]{label=(\Roman*)}
		\setlist[enumerate,7]{label=\arabic*}
		\setlist[enumerate,8]{label=\alph*}
		\setlist[enumerate,9]{label=\roman*}

\renewlist{itemize}{itemize}{9}
		\setlist[itemize]{label=$\cdot$}
		\setlist[itemize,1]{label=\textbullet}
		\setlist[itemize,2]{label=$\circ$}
		\setlist[itemize,3]{label=$\ast$}
		\setlist[itemize,4]{label=$\dagger$}
		\setlist[itemize,5]{label=$\triangleright$}
		\setlist[itemize,6]{label=$\bigstar$}
		\setlist[itemize,7]{label=$\blacklozenge$}
		\setlist[itemize,8]{label=$\prime$}

\setlength{\topsep}{0pt}\setlength{\parindent}{0pt}

 %%%%%%%%%%%%  This sets linespacing (verticle gap between Lines) Default=1 %%%%%%%%%%%%%%


\renewcommand{\arraystretch}{1.3}


%%%%%%%%%%%%%%%%%%%% Document code starts here %%%%%%%%%%%%%%%%%%%%



\begin{document}
\begin{Center}
{\fontsize{36pt}{43.2pt}\selectfont Android 9.0 Pie\par}
\end{Center}\par

{\fontsize{18pt}{21.6pt}\selectfont \textbf{\uline{Pros:}}\par}\par

\setlength{\parskip}{8.04pt}
\textbf{\textcolor[HTML]{4A4A4A}{1. Gesture Navigation}}\par

\textcolor[HTML]{4A4A4A}{This is a new feature introduced by Google’s Android wherein you will not be having your home button and recents button instead it is folded into one home screen button so that the users can interact more fluidly.}\par

\textcolor[HTML]{4A4A4A}{The new Android home button reacts somewhat similar to the iPhone X style. Overall, the new gesture navigation feature is very easy to use and doesn’t take long to respond.}\par

\textbf{\textcolor[HTML]{4A4A4A}{2. The Digital Wellbeing Dashboard}}\par

\textcolor[HTML]{4A4A4A}{This feature helps users get more useful insights about how they use phone. For example, it shows how many times you wake up your phone, how much time you spend on various apps, etc. Using this information, a user can now focus on reducing the time he spends while using his phone or an app.and hence optimize his time spent on the smartphone.}\par

\textbf{\textcolor[HTML]{4A4A4A}{3. The Shush Feature}}\par

\textcolor[HTML]{4A4A4A}{Shush which is a new gesture will help users easily put their phone to $``$Do Not Disturb$"$  mode by simply flipping their phone screen down. This mode gets automatically enabled when you place your phone screen down.}\par

\textbf{\textcolor[HTML]{4A4A4A}{4. Adaptive Battery}}\par

\textcolor[HTML]{4A4A4A}{Using the latest AI techniques, the OS will be able to identify and understand which are the apps the user is most likely to use in the next few hours. Using this information, the OS can optimize the smartphone’s CPU usage which can, in turn, improve battery performance by up to 30$\%$ .}\par

\textbf{\textcolor[HTML]{4A4A4A}{5. Adaptive Brightness}}\par

\textcolor[HTML]{4A4A4A}{This feature also utilizes AI techniques to adjust the screen brightness according to usage. The screen brightness automatically adjusts to the environment and activities of the user.}\par


\vspace{\baselineskip}

\vspace{\baselineskip}
\textbf{\textcolor[HTML]{4A4A4A}{6. App Actions}}\par

\textcolor[HTML]{4A4A4A}{App actions are actually small actions or commands that trigger a certain app. Since the Android Pie hinges so much on predictability, it also has a feature that pops up the actions when the OS thinks you will need them. For example, the OS will automatically show you the music app popup when you plug in the headphones.}\par

\textbf{\textcolor[HTML]{4A4A4A}{7. Display Rotation}}\par

\textcolor[HTML]{4A4A4A}{The new advanced display rotation feature will probably be the most wildly used feature. If you have the auto rotation feature turned off, every time you rotate your phone the Android Pie will show a rotate icon. The screen will rotate if you click it, else it will disappear after a few seconds.}\par

\textbf{\textcolor[HTML]{4A4A4A}{8. New Volume $\&$  Screenshot Interface}}\par

\textcolor[HTML]{4A4A4A}{Concluding\ on the Android Pie's more notable features is a pair of small interface changes like the screenshot and volume UIs. On raising or lowering the volume, Android Pie presents a vertical slider on the right side of the phone, unlike the horizontal element in previous versions. Pressing the volume rocker now modifies media volume by default, instead of the ringer volume.  }\par

\textcolor[HTML]{4A4A4A}{Another noticeable change is the screenshot feature. On taking a screenshot the OS automatically presents options to edit it, where the user can quickly crop or make changes to the picture before saving it.}\par

\textcolor[HTML]{4A4A4A}{9. }{\fontsize{13pt}{15.6pt}\selectfont \textcolor[HTML]{4A4A4A}{It has better display of notifications}}\par}\par

{\fontsize{13pt}{15.6pt}\selectfont \textcolor[HTML]{4A4A4A}{\parbox{\linewidth}{10. It offers improved flow with more speed, It presents more customization, It has Dual Camera Support for developers, It offers privacy improvements at apps, It presents new lock options, It has New Emojis, It comes with better indoor navigation and it is easier to take a screenshot}}}\par}\par

\subsubsection*{Pros of Android Pie}
\addcontentsline{toc}{subsubsection}{Pros of Android Pie}
\begin{itemize}
	\item {\fontsize{18pt}{21.6pt}\selectfont \textbf{\textcolor[HTML]{262626}{Better UI}}\par}
\end{itemize}\par

\textcolor[HTML]{4A4A4A}{The new android comes with a unique interface which is both attractive and easy to use. Multiple small UI tweaks make this Android version quite user-friendly.}\par

\begin{itemize}
	\item \textbf{\textcolor[HTML]{4A4A4A}{Time-Saving Features}}
\end{itemize}\par

\textcolor[HTML]{4A4A4A}{The\ new app actions feature saves a lot of time by predicting and suggesting the apps you are most likely to use.  }\par

\begin{itemize}
	\item \textbf{\textcolor[HTML]{4A4A4A}{Digital Wellbeing}}
\end{itemize}\par

\textcolor[HTML]{4A4A4A}{Google has long been advocating the concept of digital wellbeing and the new dashboard, with helpful insights, makes sure that you can optimize their time spent on smartphone and improve your digital wellbeing.}\par

\begin{itemize}
	\item \textbf{\textcolor[HTML]{4A4A4A}{Better Battery Life}}
\end{itemize}\par

\textcolor[HTML]{4A4A4A}{Adaptive brightness and battery features optimize the battery consumption and noticeably improve the battery life.}\par

\textcolor[HTML]{4A4A4A}{ }\par

\subsubsection*{Cons of Android Pie}
\addcontentsline{toc}{subsubsection}{Cons of Android Pie}
\begin{itemize}
	\item \textcolor[HTML]{4A4A4A}{Gesture Feature needs some fine tuning. The new gesture feature is not completely flawless, it needs some improvements.}\par

	\item \textcolor[HTML]{4A4A4A}{Although Google has tried to improve the notifications management, it can still be overwhelming.}\par

	\item \textcolor[HTML]{4A4A4A}{A lot of features which Google advertised are currently not available. We expect them to be launched soon.}
\end{itemize}\par

\subsubsection*{A verdict on Android Pie}
\addcontentsline{toc}{subsubsection}{A verdict on Android Pie}
\textcolor[HTML]{4A4A4A}{With the new Android 9 Pie, Google has given its Operating System some really cool and intelligent features that don't feel like gimmicks and has produced a collection of tools, utilizing machine learning, to promote a healthy lifestyle. Android 9 Pie is a worthy upgrade for any Android device. Currently, it is only available for Pixel users, but it will also be launched for other devices very soon.}\par


\vspace{\baselineskip}

\printbibliography
\end{document}